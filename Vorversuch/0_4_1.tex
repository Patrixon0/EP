\documentclass[main.tex]{subfiles}

\begin{document}

\section{0.4.1}
\subsection{a}
Wir bauen eine simple Schaltung aus Oszilloskop, Funktionsgenerator, Koaxialkabel mit entsprechendem Abschluss und RC-Filter auf. Die Schaltung dazu sieht wie folgt aus:\\
%\center
%\includegraphics[]{}
\\
Wir messen einen Rechteckimpuls bei $2 MHz$ und $4 V_{pp}$. 
\subsection{b}

Wir können dem Diagramm die Anstiegszeit BLOP entnehmen.

\subsection{c}
Nun beobachten wir, wie hoch die Dämpfung durch ein vorgeschaltetes RC-Glied, also ein Tiefpassfilter, ist. \\
Die Funktionsgeneratorspannung beträgt $4 V_{pp}$
\begin{table}[]
    \centering
    \begin{tabular}{c|c}
         $Hz$ & $V_S$ \hline\\
         1e2 & 2.00 \\
         1e3 & 1.96 \\
         3e3 & 1.91 \\
         1e4 & 1.63 \\
         2e4 & 1.15 \\
         3e4 & 0.88 \\
         5e4 & o.56 \\
         7e4 & 0.40 \\
         9e4 & 0.32 \\
         1e5 & 0.32 \\
         1.5e5 & 0.20 \\
         2e5 & 0.16 \\
         2.5e5 & 0.12 \\
         3.5e5 & 0.09 \\
         7.5e5 & 0.04
         
    \end{tabular}
    \caption{Caption}
    \label{tab:my_label}
\end{table}
\end{document}