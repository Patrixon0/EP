\section{Durchführung}

\subsection{Versuchsaufgabe 1: Statische Messung der Diodenkennlinie}

Zur statischen Messung der Diodenkennlinie ($I = f(U)$) wird die Diode in Reihe mit einem Widerstand $R = 100 \si{\ohm}$ geschaltet.
In Durchlassrichtung der Dioden wird spannungsrichtig gemessen (siehe Voraufgabe F):
Die Spannung $U$ wird dabei nur über die Diode mit dem analogen Messgerät (UNIGOR 4P) gemessen, 
der Strom $I$ hingegen mit dem digitalen Multimeter (DMM), da es etwas ungenauer ist.

In Sperrrichtung wird umgekehrt verfahren: Für die stromrichtige Messung wird die Spannung $U$ über Diode und analogem Amperemeter gemessen,
der Strom $I$ hingegen mit dem DMM.

Aus unserer Messung bei geeigneten Spannungswerten ergeben sich folgende Kennlinien:



\subsection{Versuchsaufgabe 2: Oszillogramm der Diodenkennlinie}

