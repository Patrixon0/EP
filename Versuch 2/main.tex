\documentclass[ngerman]{scrreprt}

\usepackage{pakete}

\title{Versuch 2: Diodenkennlinien}
\subtitle{EP 2025 Kurs B}
\author{Patrick Steinbach, s24pstei; Theobald Rösler, s49troes}
\date{\today}
%\addbibresource{refs.bib}

\begin{document}

%\include{erklaerung}

\maketitle
\tableofcontents
\pagenumbering{roman}
\clearpage
\pagenumbering{arabic}

%\subfile{0_1.tex}

%\listoffigures
%\listoftables
%\printbibliography[heading=bibintoc]


\section{A}
\section{B}
\section{C}
% Theo

\subsection{a}
\subsection{b}
\subsection{c}

Reihenschaltung: Der Gesamtstrom $I$ der Schaltung fließt durch Diode D und Widerstand $R$:

\[
I = \frac{U_D}{R_D} = \frac{U_R}{R} = \frac{U{R + R_D}}
\]

Die Kennlinie ist also fast wie bei der Diode, nur der Anstieg ab der Schwellenspannung ist kleiner.
\subsection{d}

Parallelschaltung: Der Gesamtstrom $I$ setzt sich aus dem Strom durch die Diode $I_D$ und den Strom durch den Widerstand $I_R$ zusammen:

\[
I = I_D + I_R = \frac{U_D}{R_D} + \frac{U_R}{R} = \frac{U_D}{R_D} + \frac{U - U_D}{R}
\]

Die einzelnen Kennlinien von $R$ und $D$ werden also lediglich addiert.


\subsection{e}
\subsection{f}


\section{D}
% Theo
\section{E}
\section{F}
\section{G}
\section{H}
\section{I}
% Theo
\section{J}
% Theo
Es handelt sich um eine Spannungsteilerschaltung:

\[
U' = U_0 \cdot \frac{R_L}{R + R_L}
\]

Extremwerte:

\begin{itemize}
    \item $R_L \rightarrow 0 \Rightarrow U' \rightarrow 0$
    \item $R_L \rightarrow \infty \Rightarrow U' \rightarrow U_0$
\end{itemize}


\section{K}
\end{document}